\documentclass[10pt,letter]{article}
\usepackage{graphicx}
\graphicspath{{design-figs/}}
\usepackage{fullpage}
\usepackage{multirow}
\usepackage{array}

%
%  The plots in this document are produced by the script
%     examples/designdoc.py
%  And use a Subaru exposure, visit 108792 ccd 5.
%
%  To run/rerun this, see the NOTES file, and for the plot
%  command-lines, see plots.sh
%


% Add used figures to git:

% for x in $(grep 'use .*pdf' design.log | sed 's/<use //g' | tr '>' ' '); do git add $x; done

\author{Dustin Lang}
\title{Development of a Deblender}

\newcommand{\figref}[1]{Figure~\ref{#1}}
\newlength{\colw}
\setlength{\colw}{0.12\textheight}

\newcolumntype{C}{>{\centering\arraybackslash}m{\colw}@{}}
\newcommand{\mcol}[2]{\multicolumn{#1}{>{\centering\arraybackslash}m{#1\colw}@{}}{#2}}

\begin{document}
\maketitle

\section{Introduction}

This document describes the iterative design of a deblender through
experiments.  It is presented as a sequence of refinements guided by
looking at the results on real data.


%  hg clone hsc-gw2.mtk.nao.ac.jp://ana/hgrepo/supaDb
%  python bin/supa_ingest.py ham.sql3 data/SUP_20{09,10}.txt
%  (echo "# id filter ra decl expTime type mode object"; sqlite3 ../supaDb/ham.sql3 "select id,filter,ra,decl,expTime,type,mode,object from suprimecam where type='OBJECT' and exptime>119 and filter like 'W%' and mode like 'IMAG%';") > ham.txt
%  python select.py
%
% obj SDSS1115 : 79 exposures
% Filter W-S-G+ with 4 exposures, total exptime 1200.0 ( [ 300.] )
% Filter W-S-I+ with 8 exposures, total exptime 1920.0 ( [ 240.] )
% Filter W-S-R+ with 7 exposures, total exptime 2100.0 ( [ 300.] )
% Filter W-S-Z+ with 60 exposures, total exptime 7200.0 ( [ 120.] )
% 
% Arbitrarily choose visit 108792 -- G-band


\section{Starting ansatz: The SDSS deblender}

The initial ansatz,\footnote{%
  \emph{ansatz:} an educated guess that is verified later by its results (Wikipedia).}
 due to Robert Lupton \cite{rhldeblend} and
implemented in the SDSS \emph{Photo} software, is that astronomical
sources have two-fold rotational symmetry and a peak in the center.
This is a fiendishly clever observation that will take us a long way.
We will begin by describing the SDSS deblending algorithm built around
this ansatz, sweeping several details under the rug until later.

The SDSS deblender takes as input a ``footprint'' and a set of
``peaks''.  A footprint is a connected set of pixels that are
significantly above a detection threshold, grown by a margin of about
the size of the PSF.  ``Peaks'' are just what you expect:
(PSF-convolved) pixels significantly higher than their neighbors.

For each peak, the SDSS deblender builds a ``template'' image by
applying the symmetry ansatz.  Starting from the peak pixel, at
position $(p_x,p_y)$, the template at positions $(p_x + dx, p_y +
dy)$, $(p_x - dx, p_y - dy)$ contains the \emph{minimum} of those two
pixel values.  The difference between the minimum and the value at the
higher pixel is presumed to be due to other blended sources.

Having built a template for each peak, the SDSS deblender computes the
least-squares \emph{weight} for each template to best reproduce the
observed image.

Finally, the counts in each image pixel are split between the
templates covering that pixel in proportion to the values of the
templates.

This process is illustrated in \figref{fig:sdss1}.

\begin{figure}[p]
\newcommand{\exfig}[1]{\includegraphics[width=0.12\textheight]{design-sdss-#1}}
\begin{center}
\begin{tabular}{m{1.5in}CCC}
  Original image& & \exfig{image} \\
  Parent footprint& & \exfig{parent} \\
  Symmetric templates& \exfig{t0} & \exfig{t1} & \exfig{t2} \\
  Weighted templates& \exfig{tw0} & \exfig{tw1} & \exfig{tw2} \\
  Sum of weighted templates& & \exfig{tsum} \\
  Fraction of flux apportioned to each child (black=0\%, white=100\%)& \exfig{f0} & \exfig{f1} & \exfig{f2} \\
  Deblended children & \exfig{h0} & \exfig{h1} & \exfig{h2} \\
\end{tabular}
\end{center}
\caption{A sketch of the SDSS deblender, with many tricks omitted.
  Symmetric templates are built for each peak and the templates are
  weighted to best reproduce the original image, in a least-squares
  sense.  The flux is split in proportion to the weighted
  templates.\label{fig:sdss1}}
\end{figure}



\clearpage

\section{Tweak: Monotonic templates}


See \figref{fig:mono1} for an example of the ``three-in-a-row''
effect, where three sources in a line cause the middle template to
pick up flux from its siblings.


We can attempt to remedy this problem by adding a term to the ansatz:
we expect a galaxy profile to have a central peak, twofold rotational
symmetry, and a \emph{monotonically decreasing} profile.  We produce
the template as before, then apply the \emph{monotonic} constraint by
starting from the center pixel and ``casting the shadow'' of each
pixel and ensuring that all pixels in the shadow have value at most
that of the shadowing pixel.


While the monotonic constraint largely solves the ``three-in-a-row''
problem, it introduces strong ``ray'' artifacts into the template.
These are due to negative-going noisy pixels, which cast long shadows
on the template.  One might worry that these artifacts would introduce
biases in the galaxy shape measurements.




\begin{figure}[p]
\setlength{\colw}{0.12\textheight}
\newcommand{\exfig}[1]{\includegraphics[width=\colw]{design-mono1-#1}}
\begin{center}
\begin{tabular}{m{1in}CCC}
  Parent footprint & & \exfig{parent} \\
  Weighted templates & \exfig{tw0} & \exfig{tw1} & \exfig{tw2} \\
  Flux fractions     & \exfig{f0} & \exfig{f1} & \exfig{f2} \\
  Deblended children & \exfig{h0} & \exfig{h1} & \exfig{h2} \\
\end{tabular}
%
\vspace{3em}
%
\setlength{\colw}{0.12\textheight}
\renewcommand{\exfig}[1]{\includegraphics[height=\colw,angle=90]{design-mono3-#1}}
\begin{tabular}{m{1in}CCCC}
  Parent footprint & \mcol{4}{\exfig{parent}} \\
  Weighted templates & \exfig{tw0} & \exfig{tw1} & \exfig{tw2} & \exfig{tw3} \\
  Flux fractions     & \exfig{f0} & \exfig{f1} & \exfig{f2}    & \exfig{f3} \\
  Deblended children & \exfig{h0} & \exfig{h1} & \exfig{h2}    & \exfig{h3} \\
\end{tabular}
\end{center}
\caption{Two examples of the ``three-in-a-row'' failure mode: when
  three sources are aligned, the middle template can have a large
  value near its siblings' peaks since they are arranged symmetrically
  around the middle peak.  The result is that the middle peak
  ``steals'' flux from its siblings.\label{fig:mono1}}
\end{figure}




\begin{figure}[p]
\begin{center}
\newcommand{\exfig}[1]{\includegraphics[width=0.12\textheight]{design-mono2-#1}}
\newcommand{\befig}[1]{\includegraphics[width=0.12\textheight]{design-mono1-#1}}
\begin{tabular}{m{1in}CCC}
  Parent footprint & & \exfig{parent} & \\
  Weighted templates (before) & \befig{tw0} & \befig{tw1} & \befig{tw2} \\
  Weighted templates & \exfig{tw0} & \exfig{tw1} & \exfig{tw2} \\
  Flux fractions     & \exfig{f0} & \exfig{f1} & \exfig{f2}    \\
  Deblended children (before) & \befig{h0} & \befig{h1} & \befig{h2} \\
  Deblended children & \exfig{h0} & \exfig{h1} & \exfig{h2}   \\
\end{tabular}
\end{center}
\caption{Deblending results using monotonic templates for the example shown in
  \figref{fig:mono1} (top).\label{fig:mono2}}
\end{figure}

\begin{figure}[p]
\begin{center}
\newcommand{\exfig}[1]{\includegraphics[height=0.12\textheight,angle=90]{design-mono4-#1}}
\newcommand{\befig}[1]{\includegraphics[height=0.12\textheight,angle=90]{design-mono3-#1}}
\begin{tabular}{m{1in}CCCC}
  Parent footprint & \mcol{4}{\exfig{parent}} \\
  Weighted templates (before) & \befig{tw0} & \befig{tw1} & \befig{tw2} & \befig{tw3} \\
  Weighted templates & \exfig{tw0} & \exfig{tw1} & \exfig{tw2} & \exfig{tw3} \\
  Flux fractions     & \exfig{f0} & \exfig{f1} & \exfig{f2}    & \exfig{f3} \\
  Deblended children (before) & \befig{h0} & \befig{h1} & \befig{h2}    & \befig{h3} \\
  Deblended children & \exfig{h0} & \exfig{h1} & \exfig{h2}    & \exfig{h3} \\
\end{tabular}
\end{center}
\caption{Deblending results using monotonic templates for the example shown in
  \figref{fig:mono1} (bottom).\label{fig:mono3}}
\end{figure}


\clearpage

\section{Tweak: Median filtered templates}


One strategy for countering the ``ray'' artifacts is to reduce the
number and strength of negative-going noisy pixels.  One option is to
apply a median-filter to the template before applying the monotonic
constraint.





\begin{figure}[p]
\begin{center}
\newcommand{\exfig}[1]{\includegraphics[width=0.12\textheight]{design-med1-#1}}
\newcommand{\befig}[1]{\includegraphics[width=0.12\textheight]{design-mono2-#1}}
\begin{tabular}{m{1in}CCC}
  Parent footprint & & \exfig{parent} & \\
  Weighted templates (before) & \befig{tw0} & \befig{tw1} & \befig{tw2} \\
  Weighted templates & \exfig{tw0} & \exfig{tw1} & \exfig{tw2} \\
  Flux fractions     & \exfig{f0} & \exfig{f1} & \exfig{f2}    \\
  Deblended children (before) & \befig{h0} & \befig{h1} & \befig{h2} \\
  Deblended children & \exfig{h0} & \exfig{h1} & \exfig{h2}   \\
\end{tabular}
\end{center}
\caption{Deblending results using median-filtered monotonic templates for the example shown in
  \figref{fig:mono1} (top).\label{fig:mono2}}
\end{figure}

\begin{figure}[p]
\begin{center}
\newcommand{\exfig}[1]{\includegraphics[height=0.12\textheight,angle=90]{design-med2-#1}}
\newcommand{\befig}[1]{\includegraphics[height=0.12\textheight,angle=90]{design-mono4-#1}}
\begin{tabular}{m{1in}CCCC}
  Parent footprint & \mcol{4}{\exfig{parent}} \\
  Weighted templates (before) & \befig{tw0} & \befig{tw1} & \befig{tw2} & \befig{tw3} \\
  Weighted templates & \exfig{tw0} & \exfig{tw1} & \exfig{tw2} & \exfig{tw3} \\
  Flux fractions     & \exfig{f0} & \exfig{f1} & \exfig{f2}    & \exfig{f3} \\
  Deblended children (before) & \befig{h0} & \befig{h1} & \befig{h2}    & \befig{h3} \\
  Deblended children & \exfig{h0} & \exfig{h1} & \exfig{h2}    & \exfig{h3} \\
\end{tabular}
\end{center}
\caption{Deblending results using median-filtered monotonic templates for the example shown in
  \figref{fig:mono1} (bottom).\label{fig:mono3}}
\end{figure}








\end{document}

