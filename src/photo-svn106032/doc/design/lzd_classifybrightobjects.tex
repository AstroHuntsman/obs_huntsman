\def \bom{{\bf Classify Bright Objects\ }}       
 
\subsection {Introduction}
 
The \bom module classifies the objects detected by find bright objects.
Classification is based on the parameters measured by measure bright objects.
The goal of this module is two fold, to identify the morphology of
an object and to determine which objects can be subtracted from a frame.
All objects that are not subtracted are reclassified by the faint
object classification.
 
\subsection {InitModule}
 
The initialization module sets the parameters required by the main
module for measuring the photometric measures of the bright objects. 
 
The {\bf InitModule} reads parameters from the ``parameters file''
into memory which is private to the module.  
 
\begin{itemize}
 
  \item {\bf boc\_minextent} : The ratio of the semi-minor axis and 
  width of the PSF (given the peak pixel within an object) below which
  the object is classified as a defect.
 
  \item {\bf boc\_psffit} : the $\chi^2$ of the fit to the PSF below which
  an object is defined as a star.
 
\end{itemize}
 
\subsection{Input}
 
The \bom module requires the following inputs,
 
\begin{itemize}
 
\item A list of OBJC (OBJECT1s in all colors) for all bright objects.
 
\end{itemize}
 
\subsection{Output}
 
On completion the \bom module outputs for each color frame,
 
\begin{itemize}
 
\item A list of OBJC structures. Each OBJC has a classification
defined by the measure objects parameters.
 
\end{itemize}
 
\subsection{Modules and Algorithms}
 
Classifications are undertaken in a modular way, each object is passed to a
module designed to identify a particular type of object (e.g.\ star, 
defect, trail). If a star passes the identification criteria for more than
one of these modules it is defined as BRIGHT\_NTYPE (i.e.\ multiple
classification). All classifications are done independently in each color. 
Possible classifications are :
 
\begin{tabular}{ll}
BRIGHT\_UNKNOWN    & Classification not possible \\
BRIGHT\_DEFECT     & Defect             \\
BRIGHT\_STAR       & Single, isolated star       \\
BRIGHT\_GHOST      & Ghost from bright star      \\
BRIGHT\_MSTAR      & Multiple star               \\
BRIGHT\_TRAIL      & Satellite trail             \\
BRIGHT\_CR         & Cosmic Ray \\
BRIGHT\_NTYPE      & Multiple classification     \\
\end{tabular}
 
\noindent The modules called by \bom are,
 
\begin{itemize}
 
\item id\_defect : Defects are identified based on their extent compared to
a DGPSF (cosmic rays, bad columns will not have a shape characteristic of
the optics and atmospheric diffraction). The semi-minor axis of each
object (the minimum extent) derived from the 2nd moments is compared with the
width of a DGPSF with amplitude the value of the peak pixel. If the semi-minor 
axis is less than {\bf boc\_minextent} times the expected DGPSF width the
object is classified as a defect.
 
\item id\_star : Classification of an object as a star (single, isolated) is 
based simply on the ability to fit a PSF to the object. If the $\chi^2$
of the fit is less than {\bf boc\_psffit} the object is classified as a star.
 
 
\end{itemize}
