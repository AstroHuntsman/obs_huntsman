\subsection{Pipelines}

	The photometric pipeline has been divided into two sub-pipelines:
a postage stamp pipeline and a frame pipeline.  The postage stamp pipeline
will act on the postage stamps from the photometric camera, the quartile
data, and the output of the astrometric and monitor telescope pipelines.
it is responsible for rough photometric and astrometric calibrations,
determining the point spread function(s), and creating the flatfields.
After this sub-pipeline is run, the results will be evaluated to determine
whether they are reasonable.  If they are (the zero points haven't changed
tremendously, for example), then we start the frame pipeline process.
The frame pipeline takes the output of the postage stamp pipeline and the
five color images from a run and produces the catalog of found objects,
among other things.  Both of these sub-pipelines operate on one scan line
from one run at a time.  Each of these sub-pipelines has one preprocessor,
one processor, and one postprocessor which run in series.

\subsection{Preprocessor}

	The photometric pipelines most naturally operate on units of
complete scan lines (one continuous data stream (run) from one column
of CCD's).  However, the data from the telescope will be written
in sets consisting of an entire night of data.  These data will be
archived into the survey data base.  The preprocessor
will query the data base and generate all of the (ascii) input files
for the processor, including the list of images in the complete
scan line and the photometric parameters to be applied to that
data set.

\subsection{Processor}

	The processor takes the output of the preprocessor plus the
raw data from a night's observing.  In the case of the postage stamp
pipeline, it generates rough astrometric calibrations, photometric
calibrations, flatfields, and the point spread function.  The frame
pipeline processor generates corrected image frames,
object lists, object parameters, and quality assurance data.  The processor 
should be capable of reading the raw data from tape or disk.  The frame
processor, since it operates on each frame independently, should be 
capable of starting and finishing in the middle of a run.

\subsection{Postprocessor}

	The level zero processor will output lists of objects and other output
as flat files.  The postprocessor will take this data and load it into the
database (after inspection of the quality assurance data from the processor).

In the case of the frame pipeline, the flatfiles will be
generated on a frame by frame basis, which will produce object
duplication in the overlap regions. So, the frame pipeline postprocessor 
will attempt to join 
the output before adding it to the database.  The overlaps will be between
scanlines as well as between frames in a scan line.

\subsection{Data flow in the Postage Stamp Pipeline Processor}

	The main program of the processor is a TCL script which accomplishes
the tasks in the accompanying figure.  The diagram shows only the scientific
steps performed; the frameworkk may opt to parallel process certain steps
(for example double buffering during read in).

\epsfxsize=0.98\textwidth
\epsfbox{psflow.eps}

\subsection{Input to the Postage Stamp Processor}

The level zero photometric pipeline will read its input as a set of ascii
files and a set of tapes.  The inputs are fully distinguished by the
process and creation timestamps, the TAI date of the observations
processed, the run number, and the ccd column number.  These quantities
are included as the header of every input file.

\subsubsection{Data Processing Plan}

	The data processing plan identifies the other files to be used in
the processing, a list of data tapes required, and a resource plan (tape
devices, disk directories, etc.) as well as the mode of operation (this is
where we can specify debugging, test overnight, use only two colors, or
normal mode, for example).  The processing plan is identified by the date
it was produced, the TAI date of the observations, the run number, and the 
column number.
The plan should contain:
\begin{itemize}
\item creationdate - date that this file is created
\sitem DATE - TAI date at start of the observing run
\sitem RUN - run number
\sitem ccdcol - camera column number
\sitem name of quartile tape(s)
\sitem list of parameter files
\sitem ncolor - number of colors to be processed
\sitem color\# - color1 might be R, color2 might be I, etc.
\sitem refcolor - which color is used for the astrometric calibration
\end{itemize}

\subsubsection{Software Parameters - as many files as necessary}

The science modules are allowed access files of parameters, templates, etc.
These files can be private to the module in the sense that no other piece
of the pipeline depends on them, and the module writer is responsible for
opening and closing the file in the init/finiModule code.  The necessity
of these files, however, must be above-board, since changes to these files
affect the scientific results, so the input files must be archived.  The
names of these files must be listed in the Data Processing Plan.  The format
of these files will be:

\begin{itemize}
\item timestamp - the date and time the file was created
\sitem process-stamp - the date and time the process was begun
\sitem DATE - TAI date at start of the observing run
\sitem RUN - run number
\sitem ccdcol - camera column number
\sitem {\it put your stuff here}
\end{itemize}

\subsubsection{Output from the MT pipeline}

Along with the standard header, this includes the extinction as a function
of time as well as any secondary photometric standards that have been
observed.

\subsubsection{Output from the astrometric pipeline}

This is a list of secondary astrometric standard stars.

\subsubsection{Input tape characteristics} 
\begin{itemize}
\item timestamp - date and time file was created
\sitem process-stamp - the date and time the process was begun
\sitem DATE - TAI date at start of the observing run
\sitem RUN - run number
\sitem ccdcol - camera column number
\sitem rawnrow - number of rows in the raw frames
\sitem rawncol - number of columns in the raw frames
\sitem nquart - number of sets of quartiles per frame
\sitem ndarks - number of dark quartile frames
\sitem nframes - number of frames in the complete scan line
\sitem run number, tape name or disk path of dark quartile  data, tape
 position of dark quartile data, one per line
\sitem run number, tape name or disk path of raw image, tape position of
 raw image,tape name or disk path of quartile data, tape position of
 quartile data, one per line
\end{itemize}

\subsubsection{Postage Stamp Data Tape}

	The online system will produce one extra data tape per run.  The
tape will contain the quartile data used in flatfielding and the postage
stamps of bright stars used to find the point spread function.  If we
decide we want to use it, it will also contain the list of stars from their
postage stamps and their characteristics.  The overall
format of the tape is as follows:
\begin{enumerate}
\item a header file which says how many quartile files follow it.  This 
header will be repeated as the second file.
\item the quartiles written as FITS files.
\item a double header which says how many files of dark quartiles follow it
\item the dark quartiles (if any) written as FITS files
\item a double header which says how many files of postage stamps follow it 
\item the files containing the postage stamps, one file per frame from the online.
\end{enumerate}
There will be 30 such sets of data on the tape, one from each
of the imaging CCD's.  The beginning and end of runs will be signaled
as with the imaging data, as a set of 3 blank frames with unusual headers.

	The online system at the telescope will produce a set of quartiles
every $n$ rows, where $n \sim 1000$.  The data will be stored as FITS files, 
one for each quartile data point.  The reason the online will not concatenate
the quartiles together is so that the files can be made available for
monitoring purposes during the night without interfering with the online
operations.  The file will be 3 rows long and 2128
columns wide.  The zeroth, first, and second rows will contain the
25th, 50th, and 75th percentiles, respectively.  This will produce (2128)(3) 
scaled, 4-byte integers each $n$ rows of raw data.  The header of each 
region will contain at least:
\begin{itemize}
\item SIMPLE
\sitem BITPIX - bits per pixel, must be 32
\sitem NAXIS - number of axes, must be 2
\sitem NAXIS1 - number of columns
\sitem NAXIS2 - number of columns
\sitem BITSHIFT - scale factor
\sitem DATE - TAI date at start of the observing run
\sitem TAI - TAI time at which the frame was taken
\sitem RUN - unique number assigned to one continuous uninterupted drift scan
\sitem CCDROW - CCD camera row, 0-4
\sitem CCDCOL - CCD camera column, 0-5
\sitem CCDIDENT - serial number of the specific CCD chip used
\sitem If we are generating one quartile per frame, then:
\begin{itemize}
\sitem FRAME - number of the frame with which the quartile is associated
\end{itemize}
\sitem otherwise:
\begin{itemize}
\sitem ROWNUM - row number within the run
\end{itemize}
\sitem DATATYPE - IMAGEQUARTILE
\end{itemize}

	If we generated data from a dark readout of the CCD's, this
data will be written similarly to the imaging
quartile data.  These dark quartiles can be used to calculate the baseline
(dark) vector.  The headers will be the same as the imaging quartile data
except the DATATYPE is DARKQUARTILE.

	The files of postage stamps will be written in a FITS-ish format.
The data type of the pixels will be unsigned short integer.  The file
will be organized as a header, which is a copy of the header used for the
image frame with the following exceptions:
\begin{itemize}
\item SIMPLE = F
\sitem NAXIS = 3
\sitem NAXIS1 = number of columns in each postage stamp
\sitem NAXIS2 = number of rows in each postage stamp (usually NAXIS1+1)
\sitem NAXIS3 = number of postage stamps in the file
\end{itemize}
The first ``row" of each postage stamp will contain two 2-byte pixels 
containing the column and
row indices of the center of each postage stamp followed by a string of
zeros to complete the row.  NAXIS1 should
be odd an odd number so that there is a will-defined central pixel.

	If we put in the star lists, they will go here at the end.

\subsection{Scientific output from the Postage Stamp Pipeline Processor}

\subsubsection{Corrected mode array}

Since the flatfield vector is calculated separately for each row, we will
store only the corrected mode array from which the flatfield vector can be
calculated.  This will be output as regions 2048 (or 2046?) columns by
about 1000 rows,
depending on the length of the night, as scaled long integers.  The header
will be the same as the quartile images, except that the DATATYPE is
CORRECTEDMODEARRAY.  There will be one corrected mode array for every color.

\subsubsection{Photometric parameters}

One number per frame that converts counts to Janskys.

\subsubsection{Astrometric parameters}

Six linear coefficients per frame per color.  All but the R coefficients
translate row,col in the color into R.  The R coefficients take row,col
in the run and translate it to great circle coordinates.

\subsubsection{Point Spread Function}

A set of parameters which describe the psf, one set per frame

\subsubsection{One Sky value per frame}

\subsection{Quality assurance output}

\subsubsection{Processing Report}

The processing report must include at least the following information:
\begin{itemize}
\item timestamp - date and time file was created
\sitem process-stamp - the date and time the process was begun
\sitem DATE - TAI date at start of the observing run
\sitem RUN - run number
\sitem ccdcol - camera column number
\sitem any errors or warnings that occurred
\sitem number and designation of images processed
\sitem input files used
\sitem output files generated
\end{itemize}

\subsection{Data flow in the Frame Pipeline Processor}

	The main program of the processor is a TCL script which accomplishes
the tasks in the accompanying figure.  This main TCL script will control
which tasks are performed on which CPU, and in general can only be restarted 
at a new five color frame.  The diagram shows only the scientific steps
performed; the framework may opt to parallel process certain steps (for
example double buffering during read in).

\epsfxsize=0.98\textwidth
\vspace{-1.5cm}
\epsfbox{ppflow.eps}

\subsection{Input to the Frame Pipeline Processor}

	The level zero photometric pipeline will read its input as a
set of ascii files and a set of tapes.  The input includes the data
processing plan, the software parameters, the hardware parameters,
the list of input images and quartiles, the raw data tapes, the quartile
tape, and the catalog of bright stars and galaxies.  The inputs are fully
distinguished by the process and creation timestamps, the TAI date of 
the observations to be processed, the run number, and the ccd column 
number.  These quantities are included as the header of every input file.

\subsubsection{Data Processing Plan}

	The data processing plan identifies the other files to be used in
the processing, a list of data tapes required, and a resource plan (tape
devices, disk directories, etc.) as well as the mode of operation (this is
where we can specify debugging, test overnight, use only two colors, or
normal mode, for example).  The processing plan is identified by the date
it was produced, the TAI date of the observations, the run number, and the 
column number.
The plan should contain:
\begin{itemize}
\item creationdate - date that this file is created
\sitem DATE - TAI date at start of the observing run
\sitem RUN - run number
\sitem ccdcol - camera column number
\sitem list of raw tapes
\sitem list of parameter files
\sitem rawdevtype - tape or disk
\sitem rawdev - disk name or tape name for reading raw frames
\sitem quartdevtype - tape or disk
\sitem quartdev - disk name or tape name for reading quartiles
\sitem corrdev - disk name or tape name for writing corrected frames
\sitem stampdev - disk namePostage Stamp or tape name for writing postage stamps
\sitem listdev - disk name or tape name for writing catalogs
\sitem brightstarfile - name of file into which to write bright star list
\sitem catalogfile - name of file into which to write catalog of objects
\sitem ncolor - number of colors to be processed
\sitem color\# - color1 might be R, color2 might be I, etc.
\sitem refcolor - which color is used for the astrometric calibration
\sitem startframe - frame to start processing
\sitem endframe - frame to end processing, frames will repeat if endframe is greater than nframe
\sitem verbose - gives verbose output if set to true
\end{itemize}

\subsubsection{Software Parameters - as many files as necessary}

The science modules are allowed access files of parameters, templates, etc.
These files can be private to the module in the sense that no other piece
of the pipeline depends on them, and the module writer is responsible for
opening and closing the file in the init/finiModule code.  The necessity
of these files, however, must be above-board, since changes to these files
affect the scientific results, so the input files must be archived.  The
names of these files must be listed in the Data Processing Plan.  The format
of these files will be:

\begin{itemize}
\item timestamp - the date and time the file was created
\sitem process-stamp - the date and time the process was begun
\sitem DATE - TAI date at start of the observing run
\sitem RUN - run number
\sitem ccdcol - camera column number
\sitem {\it put your stuff here}
\end{itemize}

\subsubsection{Hardware Parameters}

\begin{itemize}
\item timestamp - date and time file was created
\sitem process-stamp - the date and time the process was begun
\sitem DATE - TAI date at start of the observing run
\sitem RUN - run number
\sitem ccdcol - camera column number
\sitem Parameters for each CCD:
\begin{itemize}
\item ccdrow - camera row
\mitem ccdident - CCD serial number
\mitem overscanleft - the number of overscan columns on the left side
\mitem overscanright - the number of overscan columns on the right side
\mitem x offset relative to instrument center
\mitem y offset relative to instrument center
\mitem CCD rotation relative to instrument center
\mitem x scale factor correction to nominal (should be 1 forever)
\mitem x distortion constant (should be 0)
\mitem y distortion constant (should be 0 but may not be)
\mitem readout noise
\mitem gain (photons/count)
\mitem pixel scale, arcseconds
\mitem nominal photometric calibration constant
\mitem catalog of column defects with  types for each CCD in the column
\end{itemize}
\end{itemize}

\subsubsection{Calibration Parameters}

The output of the postage stamp pipeline: photmetric parameters, astrometric
parameters, psf, sky value.  All quantities are one per frame.

\subsubsection{Input Image Files} 
\begin{itemize}
\item timestamp - date and time file was created
\sitem process-stamp - the date and time the process was begun
\sitem DATE - TAI date at start of the observing run
\sitem RUN - run number
\sitem ccdcol - camera column number
\sitem rawnrow - number of rows in the raw frames
\sitem rawncol - number of columns in the raw frames
\sitem nframes - number of frames in the complete scan line
\sitem run number, tape name or disk path of raw image, tape position of
 raw image, one per line
\end{itemize}

\subsubsection{Raw Image Tapes}

	There are six columns of photometric CCD's.  Each of the CCD's
in the column scans the same piece of the sky.  There are five CCD's in
each column (U, G, R, I, and Z).  Each tape will contain data from one 
column of CCDs, square root encoded, and written in (almost)
FITS format.  (FITS does not support 16 bit unsigned integers or square
root encoding.)  There is one physical record per frame.
A single run can span many tapes, but a tape can contain only one run.
The first ncolor frames on tape $N+1$ must be a copy of the last
ncolor frames from tape $N$.
Any information written as ``beginning of run' or ``end of run' records
must be written as frames, using the header for the information.  There
must be ncolor such records (the ReadNextFrame modules will not recognize
these records as being special).  Two consecutive EOF marks indicates
the end of the data tape.

	The CCD chip is read out into a serial register which is 40 columns
larger than the CCD array, sticking out 20 columns on each side of the chip.
The charge is transferred into the register one row and a time and read
out through two amplifiers, one at each end of the register.  Each row
read out of each amplifier consists of 20 extended pixels (from the serial
register), 1024 pixels of real data, and then N overscan pixels (Jim
Gunn suggests N=20).  These pixels will be rearranged by the DA system
so that each FITS file has columns:
\begin{itemize}
\item 0..19 left extended pixels
\sitem 20..29 left overscan
\sitem 30..1063 pixel data from left side of chip
\sitem 1064..2087 pixels from right side of chip
\sitem 2088..2107 right overscan
\sitem 2108..2127 right extended pixels
\end{itemize}
Pixels 1064-2087, 2088-2107, and 2100-2127 will be ordered in reverse from 
the order they are read out so that the
position on the sky is continuous across the center of the frame.  The
reponse of the pixels in columns 30 and 2127 is nonlinear, so they should
be ignored during processing.  However, the modules should not make assumptions
about the frame size, pixel scale, sky level, or photometric calibration.  This
information should be passed to it from the framework.

	The distance between centers of the CCD chips is 65mm, which
corresponds to 2708.3 rows.  To write out frames that line up in
the many colors, we must write the frames in units of 2708/n rows, where
n is a small integer.  We have chosen n=2, which gives us
2107x1354 pixel frames written to tape.  These frames will be written
to tape so that the five colors of a given patch of sky will be
I, R, G, and U.  The first and second frames form the Z chip (2107x1354)
will not have corresponding data from the other frames.  The tape will
be padded by writing blank frames for the other four chips.  The third
and fourth frames from the Z chip will have corresponding data from only
the I frame.  The ninth frame from the Z chip is the first one with
corresponding data in all chips: the seventh frame from the I chip,
fifth from R, third from G, and first from U.  The tape will contain
the frames in the following order:
\begin{verbatim}
{beginning of tape/beginning of run records written as five frames}
{Z1, Ib, Rb, Gb, Ub}, {Z2, Ib, Rb, Gb, Ub},
{Z3, I1, Rb, Gb, Ub}, {Z4, I2, Rb, Gb, Ub},
{Z5, I3, R1, Gb, Ub}, {Z6, I4, R2, Gb, Ub},
{Z7, I5, R3, G1, Ub}, {Z8, I6, R4, G2, Ub},
{Z9, I7, R5, G3, U1}, {Z10,I8, R6, G4, U2},
{Z11,I9, R7, G5, U3}, {Z12,I10,R8, G6, U4}, ...,
\end{verbatim}
where Z1 means the first frame from the Z chip, Z2 the second, and so on.
I1 is the first frame from the I chip.  Ib, Fb, Gb, and Ub are the blank
frames written to pad the beginning and ending of the tape.  The five
frames withing the curly brackets arer read as a corressponding group
of frames.


	The FITS header must contain at least the following items, other
necessities will be added later:
\begin{itemize}
\item SIMPLE - F
\sitem FILETYPE - \#		/* Square root encoded
\sitem BITPIX - bits per pixel, must be 16
\sitem NAXIS - number of axes, must be 2
\sitem NAXIS1 - number of columns
\sitem NAXIS2 - number of rows
\sitem DATE - TAI date at start of the observing run
\sitem TAI - TAI time at which the frame was taken
\sitem RUN - unique number assigned to one continuous uninterupted drift scan
\sitem FRAME - number of the frame within the run, CCD row, and CCD column
\sitem CCDROW - CCD camera row, 0-4
\sitem CCDCOL - CCD camera col, 0-5 
\sitem CCDIDENT - serial number of the specific CCD chip used
\sitem DRIFTRAT - microsecs per row (linestart rate)
\sitem LINENO - frame starting line number within the run
\sitem RA - HH:MM:SS.SS right ascention of center of instrument rotator at begin
\sitem DEC - DD:MM:SS.S declination of center of instrument rotator at begin
\sitem EQUINOX - equinox of the RA and DEC
\sitem ROT - position of the instrument rotator at begin
\sitem RAEND - HH:MM:SS.SS right ascention of center of instrument rotator at end
\sitem DECEND - DD:MM:SS.S declination of center of instrument rotator at end
\sitem ROTEND - position of the instrument rotator at end
\sitem DATATYPE - IMAGE
\sitem ARCHTAPE - archive tape name
\sitem ARCHFILE - archive tape file number
\end{itemize}

\subsubsection{Catalog of Bright Stars and Galaxies}

	This is a catalog of all of the bright objects we will skip over
during processing.  The preprocessor chooses only those stars relevant to the 
current scan line.  For each object, we identify the part of the sky subtended
by an ellipse.  Stars will have major and minor axes of zero size.  We will
calculate the radius to exclude around stars from the magnitude of the star.
\begin{itemize} 
\item timestamp - date and time catalog was created
\sitem process-stamp - the date and time the process was begun
\sitem For each object in the catalog:
\begin{itemize}
\mitem object - object name
\mitem classification - classification of the object (we should have a limited number of possibilities)
\mitem RA - right ascention of center of object
\mitem DEC - declination of center of object
\mitem position error - the accuracy to which the position is known (arcseconds)
\mitem equinox - equinox or the RA and DEC
\mitem PA - position angle of object
\mitem majoraxis - size, in arcseconds of the major axis of the elipse
\mitem minoraxis - size, in arcseconds of the minor axis of the elipse
\mitem B - blue magnitude
\mitem other magnitudes if available
\end{itemize}
\end{itemize}
The format of the catalog will be the same as the catalogs used for the
astrometric pipeline.

\subsection{Scientific output from the Frame Pipeline Processor}

	The output from the pipeline is fully identified by the date and
time it was created, the date on which the run started (TAI), the run
number, and the ccd column number.  The output described here should be
sufficient to select galaxy targets for the spectroscopic survey, and to
achieve the primary science outlined in the Principles of Operation of
the Sky Survey Project.  Extras such as producing and object-subtracted
sky frame are not essential to the operation of the survey, and therefore
will not be implemented in level zero.

\subsubsection{Corrected frames}

	The corrected frames will be 1500 rows by 2048 columns (including
the overlaps), and will be written to tape with very few
defects removed.  The frames will be bias subtracted and divided by
the flatfield vector.  In addition, bad columns will be interpolated and
chip defects removed as well as possible.  Defects such as aircraft,
meteors, and satellites will not be removed in the level zero data system.
If it is deemed advantageous by the collaboration, these corrections
can be included in future versions.  If such corrections are included in
subsequent versions, we will have to supply an additional description 
of the corrected pixels, either as a bit mask or as a list.

	The corrected frames will use nearly identical headers to the
raw data.  The exceptions are that the number of columns will be smaller,
since the overscan has been clipped off, and the two keywords:
\begin{itemize}
\item CORRDATE - date the corrected frame was written to disk
\sitem STATUS - flatfielded
\end{itemize}
will be added.

\subsubsection{Bright Objects}
\begin{itemize}
\item timestamp - date and time file was created
\sitem process-stamp - the date and time the process was begun
\sitem DATE - date at start of observations for the current run (TAI) 
\sitem RUN - run number
\sitem ccdcol - camera column number
\end{itemize}

\subsubsection{Catalog of objects}

The catalog of objects will be calculated for each frame.  The results
will be concatenated to the end of a flat file after each frame.  This
catalog is the merged list including the inforation from all colors.
The file must contain at least the following information:
\begin{itemize}
\item timestamp - date and time file was created
\sitem process-stamp - the date and time the process was begun
\sitem DATE - date at start of observations for the current run (TAI) 
\sitem RUN - run number
\sitem ccdcol - camera column number
\sitem merged object lists, one per frame
\end{itemize}
This is practically the only output of the pipeline that can be made
available on disk.  All other significant outputs will have to be made
available on a slower medium.

\subsubsection{Postage Stamps}

The rasters of all of the objects found in the survey will be stored in 
all colors as ``postage stamps" (subject to the constraint that they
must cover less than 2\% of the image).  These will probably eventually be 
stored on something like a tape robot as FITS-ish files, one per frame.  
The form of these files should be identical to the form of the bright 
star files generated by the online and input to this pipeline on 
the ``additional data tape."  The plan is to
store postage stamps for all galaxies found in the survey, with the stipulation
that the total area in postage stamps cannot exceed two percent of the area
of the sky.  If the size exceeds two percent, then we don't gain much by
storing the postage stamps rather than the original images.  It is yet to
be determined how large the postage stamps should be.  They could be just
large enough to cover the sky-limited extent of the object.  This would not
allow one to determine the sky background from the postage stamp itself -
the local sky background would have to be stored in the header.

\subsubsection{Rough Sky Map}

In addition to the postage stamps of the interesting objects in the image,
we would like to store a crude reproduction of the whole image, which will
be useful for quality assurance, and as a tool for finding charts, etc.
This should be one file for each frame for each color, and should be less
than one percent the size of the original image, the smaller the better.
This could be done be binning the image in 10x10 pixel chunks.  These could
then be compressed in a lossy fashion such as square root compression.
The tools to do this will be readily available in DERVISH, so the framework
can write these out after flatfielding.

\subsection{Quality Assurance Data}

The processor must produce quality assurance data so that we can flag
substandars data before it is loaded into the database.  We list here some
obvious quality assurance data that should be generated for each run of
the photometric pipeline, and we expect that as we gain more experience with
the failure modes of the pipeline, the specification of the output will
change.

\subsubsection{Processing Report}

The processing report must include at least the following information:
\begin{itemize}
\item timestamp - date and time file was created
\sitem process-stamp - the date and time the process was begun
\sitem DATE - TAI date at start of the observing run
\sitem RUN - run number
\sitem ccdcol - camera column number
\sitem any errors or warnings that occurred
\sitem number and designation of images processed
\sitem input files used
\sitem output files generated
\end{itemize}

\subsubsection{Basic image characteristics}

	There are several obvious plots we can make to analyse the
results on a frame by frame basis.  For each frame, we will determine
the:
\begin{enumerate}
\item number of stars, galaxies, and others (three numbers)
\sitem image minimum, maximum, median, and sky in each color
\sitem psf parameters, probably $\sigma_{xx}$, $\sigma_{xy}$, and $\sigma_{yy}$
\sitem number of pixels flagged as bad.
\end{enumerate}
From this information which is written out as flat files, we can generate
a plot versus frame number for each item.

In addition, we should update three histograms after each frame.  The final
histograms will give the numbers of stars, galaxies, and ``other" as a
function of magnitude.

\subsubsection{Positions of objects detected}

	It is impossible to display the positions of the detected objects
in 10 workstation screens or less, since to do that each frame would have
an allocation of 1" tall by 1.33" wide, and displaying many hundreds of 
objects in this space would serve little purpose.
Therefore, I suggest that we calculate number counts densities for each
hundred frames in regions about 20 pixels by 20 pixels.  This would produce
ten greyscale maps that are 75x100 that will show large scale biases based
on position on the chip.

\subsection{Verbose Output from the Processor}

	If verbose mode is requested, we will get extra information
in addition to the output obtained from normal operation.
We allow ourselves here to write out more output, since it does not
have to be archived.  However, it should not be so  verbose that
it becomes useless as a debugging tool.  It is expected that each module
takes this as an argument so it can be changed at run time rather than
at compile time.  This is expected to be determined
primarily during fabrication of the pipeline, but should include things
like:

\subsubsection{Intermediate object lists}

	The object lists after FindObjects and FindBrightObjects would
be output for inspection.

\subsubsection{Masks}

	The masks are a set of 8 bits for every pixel of data.  The
designation of the bits is still under discussion, but will include
at least the following information:
\begin{itemize}
\item two bits which signify: good pixel/fixed but okay/bad pixel, don't use
\sitem one bit that says whether the pixel is part of an object
\end{itemize}

\subsubsection{Images to the screen}

	The corrected mode array, raw frames, and corrected frames could
be send to a display, and the objects as they are found could be indicated
on the corrected image.  One might want to overlay the mask of bad pixels.
The type of object found could be coded as a different shape glyph for each
object type.

\subsection{Memory/CPU Requirements}

This section will be completed when the science modules are better specified.

