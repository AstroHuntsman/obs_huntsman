\def \bom{{\bf Measure Bright Objects\ }}       
 
\subsection {Introduction}
 
The \bom module measures a set of parameters for all OBJC structures
found by the find bright objects module and merged into 5 colors in the
merge colors module. The measured parameters serve two purposes, they
must be sufficient to classify the bright objects and they must
contain all the necessary information required in the output of the
objects (i.e. bright objects that are subtracted from a frame are not
passed to the measure objects module).
 
\subsection {InitModule}
 
The initialization module sets the parameters required by the main
module for measuring the photometric measures of the bright objects. 
 
The {\bf InitModule} reads parameters from the ``parameters file''
into memory which is private to the module.  
 
\begin{itemize}
 
  \item {\bf bom\_fibreRad} : the radius of the spectroscopic fibre 
  (in pixels).
 
  \item {\bf bom\_apRad} : the radius of an aperture magnitude (in pixels).
 
  \item {\bf sb\_limit} : the surface brightness limit to which total
magnitudes are calculated.
 
  \item {\bf pixscale} : the pixel scale of the image (arcsec pixel$^{-1}$)
 
  \item {\bf bom\_outer} : the radius out to which the PSF is fitted
(in pixels).
 
  \item {\bf bom\_inner} : the radius from which the the PSF is fitted.
 
\end{itemize}
 
\subsection{Input}
 
The \bom module requires the following inputs,
 
\begin{itemize}
 
\item An array of image corrected frames of all colors.
 
\item An array of noise frames for all colors.
 
\item An estimate of the sky background across the frame.  
 
\item An estimate of the standard deviation of the sky background
across the frame.  
 
\item A list of offsets from one frame to the next.
 
\item A list of OBJC (OBJECT1s in all colors) for all bright objects.
 
\item An array of PSF structures as defined by the postage stamp pipeline.
 
\item An array of CALIB1 structures defining the photometric calibrations.
 
\item A FRAMESTAT structure describing the statistics of the color frames.
 
\end{itemize}
 
\subsection{Output}
 
On completion the \bom module outputs for each color frame,
 
\begin{itemize}
 
\item A list of OBJC structures. Each OBJC will have a set of
photometric parameters measured (independently in each color).
 
\end{itemize}
 
\subsection{Modules and Algorithms}
 
The \bom module measure a set of shape and photometric parameters for
each identified bright object. All of these parameters are measured
independently from color to color. Note that for the fibre magnitude
this is incorrect as wish to measure the light passing down the fibre
centered on the centroid of the galaxy/star measured in the r$^\prime$
band.
 
\noindent The modules called by \bom are:
 
\begin{itemize}
 
\item measure\_properties : Calculates the intensity weighted 1st and
2nd moments and the position and value of the peak pixel within the
object. From the moments it determines the centroid of the object and its
eccentricity and position angle.
 
\item aperSum : Calculates the aperture magnitude within a given
radius cantered on the intensity weighted centroid of the object.
This routine is called twice, once for an aperture the size of the
fibre ({\bf bom\_fibrerad}) and a second time for an aperture defined
by {\bf bom\_apRad}.
 
\item moTotalCounts : Calculates the total counts of an object
(defined as the light enclosed by an aperture that extends to a given
surface brightness {\bf sb\_limit}).
 
\item dgpsf\_amptoCounts : Calculates the PSF magnitude given the
amplitude the DGPSF fitted to the object.
 
\item dgpsfFitToObject1 : Fits the DGPSF defined by the postage stamp 
pipeline to an object using the intensity weighted centroids as a 
first guess. The routine returns the best fitting amplitude and centroid,
if successful, or the input centroid and a bad value (-999) if not. Fitting
is performed between the {\bf bom\_inner} and {\bf bom\_outer} radii (set
in terms of the FWHM of the core Gaussian in the PSF).
 
 
\end{itemize}
 
\subsection{Quality, Debugging, Resources}
 
The output from the measured magnitudes can be compared with the
catalog magnitudes using the tools created by the JPG group
