
\subsection{Introduction}

The monitor telescope pipeline will produce a set of atmospheric
extiction measurements for each color approximately every hour during
the night.  These will not be tied directly to runs, but will be on a
nightly basis.  In addition, the MT pipeline will usually produce a
list of stars with five color photometry in a set of calibration
patches.  However, the MT patches cover more than one scan line, and
the postage stamps from one night will on the average be taken once
every two hours on a given scan line (three calibration patches per
hour).  Therefore, we cannot be sure we will have the calibration
patches available - especially for short runs.  It is the job of the
photometric pipeline to produce one number per color per calibration
patch in the run giving the number of DN corresponding to a 20th
magnitude star as seen above the atmosphere.  There will be no
color terms.  If the calibration patches are not available, then we
will use the last zero point calibration available.  If for any reason
the calibrations found from this module are inadequate for determining
our galaxy parameters, we will discover this upon recalibration in the
database.  At that point, the run could be flagged as bad and we would
have to re-run it.

Since the stars in a calibration patch will have a range of colors,
the calibration is determined for a fiducial reference color, which is
a tunable parameter (currently set to zero). The second band used to
determine the color for each filter is also a tunable parameter.  A
line is fit to the set of colors and 20th magnitude fluxes for each
standard star-postage stamp match, and the value at the reference
color is used for the calibration.  The calibration is corrected to
zero atmosphere by multiplying by 10**[k secz / 2.5], where k is the
interpolated value for the time the patch was taken.

\subsection{Input}

\subsubsection{k(t)}

The monitor telescope must give us a set of measurements of the atmospheric
extinction and a time with which it is associated.  Each set of
measurements is an EXTINCTION struct.  This input is currently assumed to be
a list sorted by time, with all times in UT.

\subsubsection{Photometry of standard stars}

If available, we need a list of secondary standard stars in sky
coordinates (the same coordinates as the astromteric transformations) with 
an absolute photometric calibration for each one.
These should consist of several calibration patch sets grouped by
time, with calibrated AB magnitudes. We will have to experiment with
whether the magnitudes must be determined in exactly the same way as
the photometric chip magnitudes or not.  The postage stamps from the
photometric chips are only 29x29 pixels, so the sky background
determination for these stars is limited.

\subsubsection{Zero point from last time}

The calibration (number of counts (DN) for a 20th magnitude star) from
the last run successfully processed, one number per color.

\subsubsection{Photometric Postage Stamp List}

The results of analyzing the photometric postage stamps.  This is a list
of STAR1 structs, one list for each filter.  The STAR1 includes the
sky coordinates of the stars as well as the instrumental photometry.

\subsubsection{Frame information}

The module reads a list of FRAMEINFO structs, which contain the zenith
distance for each frame as well as the time at which the frame was taken.

\subsubsection{list of calibrations}

At this point the CALIB1BYFRAME structs have been partially filled, so
the list in input, and output with the mag20 and mag20Err fields set.

\subsection{Output}

\subsubsection{list of calibrations}
This is the CALIB1BYFRAME structs with the mag20 and mag20Err fields set.

\subsubsection{Calibration Patch Photometry}
The photometric parameter, corrected to zero atmospheres, for each
calibration patch.  This is not used by any other pipeline, but is required
output to the database.

\subsection{Modules}

\subsection{Find calibration patches}

This module divides the standard stars into calibration patches. The
stars in each patch should be grouped by time, with the measurements
in all bands occuring within one half hour, at roughly hourly
intervals. The module assumes the list of standard stars is sorted by
time (UT) and searches down the list, removing all stars within a set
number of seconds from the first star and putting them into the first
patch. Subsequent patches are identified by time from the first
remaining star. 

\subsubsection{Match stars}

This module finds the stars that are in common between the stars in a 
calibration patch and the postage stamp list, and extracts them.  The input 
is the two lists and the output is the list of extracted stars, with the DN
and magnitude of the stars in each color.

\subsubsection{Find patch parameter}

Use matched list to get counts-at-20th-magnitude interpolated to a
reference color for each the calibration patch.

\subsubsection{Correct patch parameter to zero atmosphere}

Use k(t) and the patch parameter to correct to zero atmospheres.

\subsubsection{Calculate photometric Parameters}

Interpolate (extrapolate) between calibration patches to get
calibration for each frame, taking into account its zenith distance
and k(t). If no patches are available, use the calibration from the
last successful run.
