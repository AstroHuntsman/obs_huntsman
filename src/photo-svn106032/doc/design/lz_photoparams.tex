
\subsection{Introduction}

The monitor telescope pipeline will produce a set of atmospheric extiction
measurements for each color approximately every hour during the night.  These
will not be tied directly to runs, but will be on a nightly basis.  In addition,
the MT pipeline will usually produce a list of stars with five color 
photometry in a set of calibration patches.  However, the MT patches cover
more than one scan line, and the postage stamps from one night will on
the average be taken once every two hours on a given scan line (three
calibration patches per hour).  Therefore, we cannot be sure we will have
the calibration patches available - especially for short runs.
It is the job of the photometric pipeline to
produce one number per color per calibration patch in the run to convert 
from DN to Janskys.  There will be no color terms.  If the calibration 
patches are not available, then we will use
the last zero point calibration available.  If for any reason the calibrations
found from this module are inadequate for determining our galaxy parameters,
we will discover this upon recalibration in the database.  At that point, the
run could be flagged as bad and we would have to re-run it.

\subsection{Input}

\subsubsection{k(t)}

The monitor telescope must give us a set of measurements of the atmospheric
extinction and a time (TAI) with which it is associated.  Each set of
measurements is one number per color which is the extinction in the
atmosphere at zenith.

\subsubsection{Photometry of standard stars}

If available, a list of secondary standard stars in great circle coordinates
with an absolute photometric calibration for each one.  These should be
organized in calibration patch sets.  For now, let's
assume these are corrected to be total integrated luminosity in Janskys.
We will have to experiment with whether the magnitudes must be determined
in exactly the same way as the photometric chip magnitudes or not.  The
postage stamps from the photometric chips are only 29x29 pixels, so the
sky background determination for these stars is limited.

\subsubsection{Zero point from last time}

The average number of Janskys/DN (for zero atmospheres) for the last run 
successfully processed, one number per color.

\subsubsection{Photometric Postage Stamp List}

The results of analyzing the photometric postage stamps.  The list will include
the great circle coordinates and the five color photometry.

\subsubsection{Reference Color}

The photometric calibration will be interpolated or extrapolated to the
Jansky/count for this input reference color.

\subsection{Output}

\subsubsection{Photometric Calibrations}
An array of photometric calibrations, one number and one time
per color per calibration patch.  If there are no calibration patches, 
the average number and time for each color
from the last successful calibration will be used.

\subsubsection{Calibration Patch Photometry}
The photometric parameter, corrected to zero atmospheres, for each
calibration patch.

\subsection{Modules}

\subsubsection{Match stars}

This module finds the stars that are in common between the stars in a 
calibration patch and the postage stamp list, and extracts them.  The input 
is the two lists and the output is the list of extracted stars, with the DN
and Janskys of the stars in each color.

\subsubsection{Calculate Jansky/count}

This module takes the matched list and produces
the Janskys/count and adds it as a column in the list.

\subsubsection{Find patch parameter}

Use matched list and reference color to get interpolated Jansky/count for
the calibration patch.

\subsubsection{Correct patch parameter to zero atmosphere}

Use k(t) and the patch parameter to correct to zero atmospheres.

\subsubsection{Calculate photometric Parameters}

Use whatever list of (row, patch parameter for zero atmosphere) data
is available to interpolate (extrapolate) one number for each frame,
which gives Jansky/count for the frame, taking into account its zenith
and k(t).

\subsection{Algorithms}

The zero atmosphere photometric parameter for a given patch is computed 
from the photometric parameter
multiplying by 10**[k secz / 2.5], where k is the
interpolated value for the time the patch was taken.

\subsection{Quality, Debugging, Resources}

The quality will be monitored by comparing the results with the
results of the last run.

This section requires very few resources.

