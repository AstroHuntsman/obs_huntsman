
\subsection{Introduction}

This module sets up the frame for faint object finding.  It must fit
the stars to the point spread function, mask and/or subtract the stars,
and extract postage stamps.
{\bf Note that the defects introduced by bright objects will
remain uncertain until the optics of the telescope system are well
understood (i.e.\ the test year). Fitting of PSF's and subtraction of
bright stars (including the extended wings of the PSF for very bright
stars) from the corrected frames will need
careful regression testing.
It is likely, therefore, that this module and {\bf Bright Objects}
will require substantial
man-power and computational resources during the Level Zero
and Level One phases.}

\subsection{Input}

\subsubsection{Five color frames}
\subsubsection{Merged bright object list}
\subsubsection{The PSF structure}
We certainly will need to use a more accurate representation
for the PSF than a simple Gaussian.  It is unclear at the present
time whether a double Gaussian, a Gaussian plus a table of
residuals (as in DAOPHOT), or some other form will be required.  Note that
the Level Zero {\it PSF} structure is relatively flexible, so
that we can make extensive tests without having to change
the pipeline framework.  The only difficult PSF in the current pipeline
structure would be putting in a table of residuals.
\subsubsection{Parameters}
A set of parameters describing the data vailue or number of sigma above
the sky above which pixels are regarded as BAD (even after subtraction
of the PPSF) due to the Poisson noise being too great to efficiently detect
underlying data.

\subsection{Output}

\subsubsection{Cleaned up five color frames}
\subsubsection{Bright object postage stamps}
\subsubsection{Updated frame masks}
\subsubsection{Updated Merged bright object list}
The stellar photometry and fitting parameters are added to the list.

\subsection{Modules}

\subsubsection{Fit stars}
For each star on the bright object list, find the fit to the psf that
best fits this object.  At the same time, do stellar photometry.
Update the Merged object list with this information.

\subsubsection{Extract postage stamps}
Extract postage stamps for all bright objects.

\subsubsection{Subtract and/or mask each bright object}

\subsection{Algorithms}

Problems that must be addressed by the algorithm:
{\it Saturated Stars:}
Saturation of stars (m$<$14) introduces two problems, fitting of the PSF to
determine the extent of a star and ghosting due to internal reflections.

\begin{itemize}
  \item Saturation : The number of saturated pixels in a star will be
kept in a temporary structure. The log area of these pixels gives
a rough estimate of the magnitude of the star and, therefore, the
probability that diffraction spikes and ghosts will require removing
(this can be tuned
during the test year). Fitting the PSF to the unsaturated region of
the star (e.g.\ from 5 $\sigma$ above the sky to 90\% of the
saturation level) the extent of the star can be determined. The scaled
PSF can then be masked or subtracted from the corrected frame. For
heavily saturated stars we require a model of the extended wings of
the PSF (this may not be implemented in the Level Zero Pipeline).

Fitting of a PSF that extends above the saturation level is simplest if
performed in floating point,
so we may need to make a temporary copy of the {\it REGION} in question
with floating-point, rather than integer, values.
As we expect $<1$  such star every 10
CCD frames (at 45$^\circ$)
this will not add significantly to the memory or overhead in the processing.

For saturated stars off the frame the diffraction spikes may still extend over
more than one frame. We will need to be able to quantify what
magnitude range this will occur for and, from the bright star catalog,
identify which frames will be affected. There will be a color
dependence in this (and all) defects where saturation/diffraction
spikes may occur in only one passband. As it is unlikely that we will
have multi--color information for all bright sources in the survey
(i.e.\ the HST Guide Star Catalog contains only b and v passbands)
it may be possible to use the Known Objects lists
only to warn of potential defects on a frame.

  \item Ghosting: In direct imaging ghosts arise from bright stars producing
internal reflections in the camera. In principle the ghosts can be mapped
during the test year as a function of position on the chip. In drift scanning
it is not clear in what form ghosting will arise. It may only be possible to
flag the presence of stars capable of inducing ghosting so the module
calculating the sky distribution can search for gradients.

For known objects, we may be able to calculate
ahead of time areas in which ghosts from bright stars are likely
to occur, which will make the task of this module lighter.
However,
it is unlikely that we will know enough about the optics of the
telescope to do this in Level Zero.

\end{itemize}

\subsection{Quality, Debugging, Resources}

