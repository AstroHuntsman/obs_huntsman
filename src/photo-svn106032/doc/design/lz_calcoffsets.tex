\newcommand{\mcm}{{\bf Calc Offsets} module}
\newcommand{\mc}{{\bf Calc Offsets}}
\newcommand{\refcolor}{{\it R}}
\renewcommand{\labelitemii}{$\circ$}
\setcounter{footnote}{0}

\subsection{Introduction}

The \mcm\ must establish a common coordinate scale for the color frames which
correspond to the same region of the sky.
This module determines a coordinate transformation
which, if applied, would map the separate color frames to a common
grid.\footnote{While this module does not actually perform an image ``warp''
with rebinning of the frame pixels, the coordinate transformation which is
determined for each color frame except \refcolor, when combined with an
interpolation kernel, is sufficient to completely specify that process.}

A scale calibration to an absolute set of astronomical
coordinates is not required at this stage.  This function is performed in
``Calculate Astrometric Parameters."
                                                             
\subsection{InitModule}

\subsubsection{Description}

The {\bf InitModule} process allocates and loads all frame-independent data
structures which are required by \mc.
For the most part, the contents of these structures are stored on disk
files.\footnote{Although there has been no discussion of how the parameters
which control the operation of the various algorithms are to be stored, I
suggest that they be in textual form -- and, therefore, editable and readable
-- and that the initialization sections accomplish the conversion to their
internal form, {\it i.e.} binary integers, binary reals, or whatever}
%An exception is the temperature and pressure environmental data which require
%computation to convert from the time sequence on the disk to an interpolation
%table structure.

\subsubsection{Input}

The input items from the disk, via the framework, are:
\begin{itemize}
  \item parameters which control the selection of bright, point-like objects as
        candidate fiducial points for the fitting procedure (see the {\bf
        Module} section for details).
%  \item the latitude and longitude of the observatory	
  \item for each color except the \refcolor\/ color, an array of transformation
        parameters which map the pixels of that CCD to the \refcolor\/ CCD
%  \item for the \refcolor\/ CCD, an array of transformation parameters which
%        map the pixels of that CCD to a uniform, equatorial coordinate system
%        with the frame centered at [0,0]
%  \item environmental data (temperature and pressure) taken during the
%        observation period
\end{itemize}

\subsubsection{Output}

Data structures containing:
\begin{itemize}
  \item all the items from the {\bf Input} section 
%except the environmental data
%  \item tables of environmental data (temperature and pressure) taken during
%        the observation run, structured for interpolation
  \item initialized statistical structures	 
\end{itemize} 

\subsubsection{Modules}

There are two identifiable modules in the level 0 design:
\begin{itemize}
  \item a main procedure which contains all of the photometric-specific code
  \item a general procedure which converts a sequence of independent/dependent
  value pairs to an interpolation table. The arguments are:
  \begin{description}
    \item[Inputs] \
    \begin{itemize}
      \item arrays of the independent and dependent variables
%\\ (time/pressure and time/temperature)
    \end{itemize}
    \item[Outputs] \
    \begin{itemize}                                                                                            
      \item arrays of interpolation coefficients 
%(temperature and pressure)
      \item status code
    \end{itemize}
  \end{description}
\end{itemize}
\subsubsection{Algorithms}

There are no unusual algorithms in the {\bf InitModule } section.
%The conversion of the environmental data to tabular form for interpolation uses
%standard methods.
\subsubsection{Quality, Debugging, Resources}

Debugging of this section is straightforward and involves confirming that the
disk file contents have been correctly transferred to the appropriate data
structures.
% and that the interpolation table correctly reproduces the run of
%temperature and pressure for the observation period.
There are no significant quality tests which can be performed on the output
structures.
The resources used by this section consist of the memory required for the
output data structures, which is small.

\subsubsection{Test Data Required}

%Test data are appropriate only for the environmental readings; and, since the
%generation of interpolation tables is performed by a general procedure, it is
%sufficient to test with data having non-uniform sample intervals.
                    
\subsubsection{Regression Testing}          

Regression testing consists of confirming that, following a coding change, the
contents of the output data structures have not changed.

\subsection{Module}
\subsubsection{Description}

The coordinate transformation function for a given frame has both static and
time-varying contributions.
The static component is determined by the mechanical placement of the CCD and
by the the possible distortions introduced by elements of the optical system.
The time-varying component is determined by to the effects of differential
refraction across the CCD and by tracking errors which accumulate between the
capture times of the individual color frames.
%At this stage, it is assumed that the gravitational distortion of the telescope
%structure as a function of zenith angle will not be significant.

There seems no reason, at this stage, to expect that the mechanical corrections
will not be adequately represented by linear equations in the two coordinates.
A linear polynomial covers the effects of translation, rotation, scaling, and
shear.
While a shearing effect seems unlikely, a rotational one could be serious since
a misalignment of the CCD pixel columns relative to the tracking trajectory
will lead to a loss of resolution transverse to the columns when a pixel-sized
element of the sky moves across one or more columns as it traverses the
CCD.\footnote{The best indication of the presence of such an effect will be an
asymmetric point spread function with the major axis perpendicular to the
tracking coordinate}
The distortions introduced by the optical system, if significant, will require
second-order terms in the
polynomial.\footnote{A pincushion effect, for example, is removed by the
quadratic terms}
Level zero will have only linear terms.

The time-dependent effects are both likely to be removed by a linear
correction.
Tracking corrections appear as a translation, in both the column and row
coordinates, between color frames of the same sky region.
The differential refraction effect, which is a function of the zenith angle,
barometic pressure, and temperature, should be removed by the linear terms as
well.

The major contribution to the transformation polynomial coefficients is
predictable and can be derived from static CCD parameters, static optical
parameters, and the time-varying zenith angle, barometric pressure, and
temperature.
The remaining corrections, which should be small, are determined by a
procedure which determines the best-fit set of coefficients which map a set
of bright, point-like, fiducial objects from all of the other frames to their
corresponding coordinates (row and column) in the \refcolor\/ frame.
\subsubsection{Input}

The input to the \mcm\ consists of:
\begin{itemize}
  \item parameters which control the selection of bright, point-like objects as
        candidate fiducial points for the fitting procedure.
% \item the latitude and longitude of the observatory	
  \item for each color except the \refcolor\/ color, an array of static
        transformation parameters which map the pixels of that CCD to the
        \refcolor\/ CCD
% \item for the \refcolor\/ CCD, an array of transformation parameters which
%        map the pixels of that CCD to a uniform, equatorial coordinate system
%       centered at the position [0,0]
  \item a list from the Make Star Lists module of the
        pixel coordinates of bright objects which lie within frame
% \item the approximate right ascension and declination of the center of the
%       \refcolor\/ frame
% \item for each color, the median Universal Time at which that frame was
%       acquired
% \item arrays of environmental data (temperature and pressure) taken during
%       the observation run, arranged for interpolation
\end{itemize}                                        

\subsubsection{Output}

The output from the \mcm\ consists of:
\begin{itemize}
  \item a list of point-like objects which were used in
        fitting the corrected transformation parameters
%  \item ch color frame, the zenith angle at which the frame was
%        taken\footnote{The zenith angle might be computed by the framework}
  \item a set of
        coordinate transformation parameters which maps the pixels of a given
        color to the \refcolor\/ strip
  \item for each color except the \refcolor\/ frame, goodness-of-fit
        measures for the coordinate transformation
%  \item for the \refcolor\/ frame, an array an array of coordinate
%        transformation parameters which maps the pixels of that frame to right
%        ascension and declination at some epoch\footnote{These parameters can
%        only be determined if some of the bright objects have been identified
%        with objects of known position}
%  \item for the \refcolor\/ frame, goodness-of-fit measures for the coordinate
%        transformation	 
  \item statistical measures which track the performance of the fitting
        algorithm
\end{itemize}

\subsubsection{Modules}
\subsubsection{Algorithms}
The algorithm consists of several steps:
\begin{enumerate}
  \item Compute the predictable, varying component of the transformation
        coefficients for each color frame and, for each frame except the
        \refcolor\/ frame, compute the coefficients which map these frames to
        the \refcolor\/ frame pixel coordinate system
  \item For each bright object within a non-\refcolor\/ frame, transform the
        pixel coordinates of that object to the \refcolor\/ pixel system using
	the coefficients from the previous step
  \item For each bright object within a non-\refcolor\/ frame and using input
        parameters from the {\bf InitModule} to control the search, find a
        matching object in the \refcolor\/ frame or reject the object
  \item Using the matched objects in each non-\refcolor\/ frame, compute a set
        of correction coefficients which minimize the position residuals when
        the objects are transformed to the \refcolor\/ frame, compute
        goodness-of-fit measures, and combine the fitted coefficients with the
        predicted coefficients
\end{enumerate}

Although it is possible, by Gram-Schmidt orthogonalization, to determine a set
of polynomials which are orthogonal on the set of fiducial objects, it is much
simpler and less time consuming to use a set of polynomials to first or
second-order\footnote{Only if observations during the engineering run indicate
that a second-order correction is necessary}
which are orthogonal within the rectangular frame.\footnote{Legendre
polynomials}
\subsubsection{Quality, Debugging, Resources}

Debugging of the {\bf Modules} section is accomplished by introducing
frame-to-frame coordinate distortions, via polynomials of the expected form
with a small random error, to several distributions of objects within the
nominal frame and confirming that the selection process correctly matches
objects from the non-\refcolor\/ frames to the \refcolor\/ frame and that the
fitting procedure correctly recovers the original distortion polynomial
coefficients within the expected errors.
The object distributions used in the debug test data should include a few
pathological cases to insure that the algorithms correctly recognize when the
fitting stage should be discarded and the predicted corrections used instead.

Both the memory and computational resources used by the \mcm\ are expected to
be small.
Most of the working memory will be from the run-time stack and only the
matching lists, which are of unpredictable (although, small) length, require
heap allocation.

\subsubsection{Regression Testing}

Regression testing consists of repeating the tests summarized in the {\bf
Debugging} section above and confirming that the results are a reasonable match
to the original test results.
Of course, if a procedure is changed, one might hope that the residuals from
the fitting procedure and the length of the list of matched objects would
improve rather than remain the same.

\subsection{FiniModule}

\subsubsection{Description}

There is very little in the {\bf FiniModule} procedure other than recording the
statistical measures to disk, via the framework, and releasing the memory
resources which were allocated in the {\bf InitModule} procedure.
\renewcommand{\labelitemii}{\pbf--}

