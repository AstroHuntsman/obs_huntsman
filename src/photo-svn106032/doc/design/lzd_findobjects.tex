\subsection{Introduction}

This module performs faint object finding on cleaned frames.
We assume that all kind of necessary corrections were done
for the cleaned frames in {\bf Correct Frames} and that
bright stars were already subtracted from them in 
{\bf Clean up Frames}. 
From the mask information contained in the region we will know which
pixels to ignore.

The main task of this module is to detect significant pixels,
to merge the detected pixels into groups which correspond to
objects, and to determine the position and the extent of each 
object. Extraction of detailed photometric parameters is
the responsibility of {\bf Measure Objects}.
Each color is processed separately.  At a later stage the information from
the five colors is merged together in {\bf Merge Colors}.
Objects are not deblended here, but in {\bf Measure Objects}.

\subsection{init\_find\_objects}

The initialization module {\bf init\_find\_objects} needs to get the values 
of parameters that control the object finding:

\subsubsection {Input}

\begin{description}
 \item[\qquad ffo\_medsize] a window size of median filter to create
the local sky frames (in units of pixels)
 \item[\qquad ffo\_peakthresh] a number to be multiplied with rms of
background to determine threshold ADU counts for peak finding
 \item[\qquad ffo\_growthresh] a number to be multiplied with rms of
background to determine threshold ADU counts for growing objects
 \item[\qquad ffo\_mode] method of determining sky background (0 = use 
rolling median; 1 = use interpolation between medians in boxes)
\end{description}

We create new regions to store the background image for each
color. These region are used for determining sky value of each objects. 
We also need to set up a temporary region to be used when convolving
an input region with Gaussian. 

\subsubsection{Output}           

Regions for background images will be the output. Of course these
don't have significant values. 

\subsection {find\_objects}

This function will find objects by searching for peaks in the PSF
convolved region. The extent of each object will be determined by
connected pixel method. Linked list of OBJECT1 structure will be
returned.

\begin{arguments}
\item{\quad region}(REGION)(I/O) contains the frame of interest.
\item{\quad noise}(REGION)(I) contains the noise frames.
\item{\quad back}(REGION)(O) holds background image.
\item{\quad template\_sigma}(double)(I) is a sigma of Gaussian which 
models PSF template
\item{\quad ffo\_medsize}(int)(I) is a window size of median
filter to create the local sky frames.
\item{\quad ffo\_peakthresh}(double)(I) a number to be multiplied
with rms of background to determine threshold ADU counts for peak
finding
\item{\quad ffo\_growthresh}(double)(I) is an isophotal threshold for
expanding object
\item{\quad frstat}(FRSTAT)(I/O) is a FRSTAT structure.
%\item{\quad -fact}(double)(I) is a factor of f-sigma rejection when
%calculating statistics on frames
%\item{\quad -ncycle}(int)(I) is a number of cycles for f-sigma
%rejection when calculating statistics on frames
%\item{\quad -thres\_obj}(double)(I) is an isophotal threshold for
%expanding object
%\item{\quad -thres\_sky}(double)(I) is an isophotal threshold for sky
%calculations
\item{\quad ffo\_mode}(int)(I) indicates a method of determining the
sky background
\item{\quad objectlists}(OBJECT1*)(O) is the list of objects found.
\end{arguments}

This function will set following values:

\begin{itemize}
\item For each object
\begin{description}
 \item[\qquad xc](float) the peak position in row (calculated by
fitting parabola).
 \item[\qquad yc](float) the peak position in col (calculated by
fitting parabola).
 \item[\qquad xmin](int) the minimum row value of connected pixels.
 \item[\qquad xmax](int) the maximum row value of connected pixels.
 \item[\qquad ymin](int) the minimum col value of connected pixels.
 \item[\qquad ymax](int) the maximum col value of connected pixels.
 \item[\qquad peak](float) the peak value.
 \item[\qquad skyLevel](float) the local background level near the object.
 \item[\qquad skySig](float) the rms of the local background level.
 \item[\qquad npix](int) the number of connected pixels of objects
above a threshold.
 \item[\qquad fiso](float) the sum of counts of pixels in objects above a threshold.
 \item[\qquad region](REGION) the extracted subregion from the
original region. The mask is copied from the mask of the original frame.
 \item[\qquad mask](MASK) set MASK\_OBJECT to the pixel which is the
part of each object. The difference of region->mask above is that mask 
for neighbor objects will not be set in this mask.
\end{description}

\item For original region
\begin{description}
 \item[\qquad mask](MASK) set MASK\_OBJECT for the pixels
corresponding found objects.
\end{description}

\item For FRSTAT structure
\begin{description}
 \item[\qquad nfaintobj](int) the number of found objects
 \item[\qquad nbadpix](int) the number of bad pixels whose mask value
is MASK\_NOTCHECKED
\end{description}
\end{itemize}

\subsubsection{Algorithms}

This module uses a hybrid scheme based on the matched filter method
and the connected pixel method. The program consists of two parts. One
is to find objects based on the matched filter method and the other is
to determine their extents based on the connected pixel method.

\paragraph{Object Finding}

The matched filter method finds the peak of the correlation between a
given image and a template. When the template has the same shape as
the object to be found, it produces the highest peak.

Stars have the same shape, at least for the first-order approximation,
and the best template for stars is the point spread function (PSF).
On the other hand, galaxies have a variety of shapes. Although there
is similarity in the profile shape for galaxies of the same
morphological type, apparent size varies according to the
distance. This means that we need a number of templates for a single
type even if we assume a common profile shape for the type.

We will use the Gaussian, which is a simplest model of the PSF, as a
template for galaxies as well as stars for the time being. The use of
different templates will be considered at a later stage. A Gaussian
profile with a standard deviation of $\sigma$ corresponding to the
seeing size is a reasonably good template for (unsaturated) stars and
very faint galaxies. Instead of having a variety of templates, we use
Gaussians with $2\sigma$, $4\sigma$, and so on as the templates for
bright galaxies. Gaussians with larger standard deviations will enable
us to find more extended objects.

In the actual application, the size of the PSF is kept unchanged while
the frame size is successively reduced by 2-by-2 binning so that the
{\it effective} size of the template gets bigger and bigger. Object
finding by this algorithm is a multi-pass process. Those objects that
are found in a pass are scissored out from the frame and put into the
catalog before the frame is binned for the next pass.  The area that
is scissored out is filled with the artificial sky background.

\paragraph{Determination of the Extent of the Object}

The peak of the correlation, that is, the position of an object can be
found by the matched filter algorithm. However, the algorithm cannot
tell us the extent of the object because we are not using a proper
template for extended objects.  It is critical to know the extent
of an object in order to scissor out the '{\it Atlas Image}', to
measure various photometric parameters, and to set mask for downstream
modules. We define the extent of an object by means of the usual
connected pixel method.  The connected pixels are searched from the
peak outward until a threshold is reached.

\subsection{Processing Flow}

Actual data processing in the find\_objects module proceeds as follows;

\begin{enumerate}
\item Take an original cleaned frame, named {\bf Original}.

\item Create a median filtered frame from {\bf Original}, which is
      named {\bf Background}.
    The size of the median window ($W_{\rm med}$, in units of pixels)
    is a tunable parameter.

\item Subtract {\bf Background} from {\bf Original}. The resulting
    frame is named {\bf Subtracted}.

\item Convolve {\bf Subtracted} with a template (currently Gaussian).
    The resulting
    frame is named {\bf Target}. The standard deviation of the Gaussian
     ($\sigma$) is a tunable parameter, but in principle $\sigma$
     should roughly correspond to the seeing size.

\item Find one-pixel peaks above a threshold in {\bf Target}. A pixel is
    regarded as a peak when its value is highest among the values of 
    eight pixels that surround the pixel.
    The threshold is $I_{peak}=f_{peak}\times\sigma_{bgd}(Target)$,
    and $f_{peak}$ is a tunable parameter.

\item Identify connected pixels associated with a peak in {\bf Target} in 
    order to obtain an initial guess of the extent of the object.
    The search is made from the peak outward until the pixel values
    reach a threshold. The threshold  is
    $I_{th1}=f_{th1}\times\sigma_{bgd}(Target)$, and
    $f_{th1}$ is a tunable parameter. 
  
\item Determine the local background level for each object. The local
    background level is the median of the values of pixels that encircle
    the object. Pixels that belong to the object are not included
    in the computation of the median. Pixel values are taken from
    {\bf Original} while the judgment whether or not a pixel belongs
    to the object is made using {\bf Target} based on  the results of
    step 6 above.
    Two parameters ($R_1$ and $R_2$) are used to define the region
    where the median is computed. The range of $X$ and $Y$ for the
    region is, $X_{min}-R_2<X<X_{min}-R_1$, 
    $X_{max}+R_1<X<X_{max}+R_2$, $Y_{min}-R_2<Y<Y_{min}-R_1$,
    $Y_{max}+R_1<Y<Y_{max}+R_2$.


\item Identify connected pixels associated with a peak in {\bf Original}.
    The search is made from the peak outward until the pixel values
    reach a threshold. The threshold is
    $I_{th2}=I_{bgd}+f_{th2}\times\sigma_{bgd}(Original)$, and
     $f_{th2}$  is a tunable parameter.

\item Patch objects in {\bf Original} by replacing the 
    value of pixels which belong to
    an object with artificial background, that is, the value of
     the local background level for each object.

\item Apply 2-by-2 binning to {\bf Original}. Then repeat from step 2
    regarding the binned frame as {\bf Original}. We run 3 cycles at
    present.

\item Merge object lists. Object lists created in cycles 
      1, 2, and 3 are 
      merged into a single object list after converting the data to 
      those with respect to the original frame. For example,
      $X$ and $Y$ must be multiplied by the binning factor.

\end{enumerate}

\subsection{fini\_find\_objects}

This function frees up all memory that {\bf find\_objects} module
required.
