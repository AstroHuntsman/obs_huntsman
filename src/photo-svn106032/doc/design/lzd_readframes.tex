The level zero design called for data decompression and reading from tape.
The level zero pipeline simply uses the DERVISH regReadAsFits verb to read
a FITS file from disk.  This same verb can be used to read from tape, but
this has not been put into the pipeline.  There is some question of whether
the data will in fact be compressed, so that operation is put off to level
one.

The only complication is that adjacent frames in the pipeline need to have 
an overlap.

Figure 1 shows a series of contiguous frames from one CCD.  The
scale is in pixels.  On the left, the frames A, B, C, D, and E
are the raw frames, and have no overlap.  When we send these frames
to be processed, we need to pad them so that objects that are on or
near the boundary between frames will be found as efficiently as
objects near the middle of a frame.
The first frame, called ``A'' here, 
is simply the size of the first frame read from tape.
Once frame ``A'' has been processed, the last few lines of it are
tagged onto frame B, making its frame sent for processing slightly larger.
This is the size of all successive frames sent for processing.

The camera design has a 42.00 mm gap between adjacent columns of CCDs,
with each CCD being 49.152 mm wide.  The overlap between the two strips
that make up a stripe is then 3.576 mm or 149 pixels.  We need to make
sure that the edge overlap between {\em strips} is at least this much.

The raw frame B on the left of Figure 1 has the last 149 lines of
the raw A frame tagged onto it before it is sent for processing
as the B frame.  All objects found in these extended frames are 
reported.

This overlap region defines the largest object (in pixels) that we will
detect and measure on one frame.  
149 is actually the {\em maximum} overlap possible.
Depending on the final alignment of the frames this number will be a few
pixels smaller in the scan direction in some cases.  Depending on the
telescope tracking, the number of overlap pixels will be smaller than
149, by some (small) amount.  We should set the maximum object size to
be comfortably smaller than this number, say 135 pixels in x and y.
Note that a highly elliptical object that is at 45 degrees can be longer
than this along the major axis, but will still be detected and measured
if it is smaller than 135 pixels in x and 135 pixels in y.

\epsfxsize=0.98\textwidth
\vskip 0.25in
\epsfbox{dist.eps}

Figure 1:  Overlap between successive frames from one CCD

