
\def \fpsf{{\bf Find PSF\ }}
\def \fbs{{\bf Find Bright Stars\ }}

\subsection {Description}

We will use the photometric chip postage stamps to determine the
point spread function(s) for the entire run.  We cannot use the stars in one
frame to define a PSF for that frame, because there will be
regions of the survey in which too few stars appear.
We will use photometric chip postage stamps {\it from the entire strip}
to compile a large set of stellar images, from which a good PSF
may be determined.  The PSF may vary as a function of time, and 
also as a function of position perpendicular to the readout direction.
All the work needed to create this model of the PSF is the job of
\fpsf .

  {\bf Note that the postage stamps will not allow us to 
model the extended wings of stars; nor can we calculate
the expected shape of the wings until the optical design is
completed.  I therefore suggest that we leave the problem of
extended wings for the Level One pipeline.}

  In addition, the \fpsf will read in the values of 
several parameters needed for the task of finding stars in the
photometric images, and perhaps information describing the 
extended wings of stars (if we decide simply to guess for
the Level Zero implementation),
and possibly allocate some memory for 
use within the module.

\subsection {Input}

  In order to compute a PSF, the module needs

\begin{itemize}
  \item the entire set of postage stamps cut out around bright objects
        in the photometric chips by the on-line Data Acquisition system.
  \item a rough model for the PSF (either a set of several parameters,
        or possibly a complete 2-D example)
  \item a set of parameters describing the acceptable limits
        for stellar candidates
\end{itemize}

Note that we {\it must} give the module some idea
for the appearance of stars, so that it will choose only actual
stellar images to define the PSF.

\subsection {Output}

  The main product of the {\bf FindPSF} is 

\begin{itemize}
  \item a model PSF, which may be dependent and position - the parameters
will be output as one complete set per frame
\end{itemize}

As by-products, we could also produce

\begin{itemize}
  \item a set of corrected postage stamps from the photometric array
  \item a list of {\it STAR1} structures containing the 
        measured properties for all the
        stars in the photometric dataset used to construct the PSF
\end{itemize}

\subsection {Modules}

  The \fpsf consists of several large, non-trivial
sub-modules, namely ones to

\begin{itemize}
  \item identify the subset of objects which are unsaturated, 
        isolated stars 
  \item build a model PSF, which may vary with position and/or time
\end{itemize}

  In addition, there is the much simpler task of

\begin{itemize}
  \item read parameter values
\end{itemize}

\subsection {Algorithms}

Identifying stars as ``good'' ones for the PSF is not
easy.  Comparing each object to the rough input PSF may not
be sufficient --- the module may have to make extra checks
to be sure that fainter objects or image defects do not occur
within some critical distance of the star.

  There are a set of utility routines in the Prototype Pipeline
for building up a model {\it PSF} structure from a set of 
regions including stars.  The model consists of a 2-D bivariate
Gaussian plus a table of residuals, identical to that used
by DAOPhot.  However, adding time and/or location dependence
will require some additional work.  We have decided that for level
zero we will only fit to a function, and not put in the residuals.
One possibility is to create a series of {\it PSF} structures
from stars which have been grouped together by time of
exposure, and then fit the Gaussian parameters with a low-order
polynomial as a function of time.  However, we will have to
experiment and see what works best.

  The {\it PSF} structure will have another additional member or two,
which describe the form of the extended wings of very bright stars
out to distances comparable to the size of an entire frame.
However, this module will not calculate that form from the
postage stamps; it will be calculated and stored in a parameter
file.

  The parameter-reading task does not
require any special algorithms.

\subsection {Quality, Debugging, Resources}

  Developing the \fpsf will require the ability
to display a postage stamp together with the model PSF
based upon it (and, if possible, the analytic approximation
and residuals separately).  Tools to graph the fitted
PSF to an actual stellar profile will be very useful.

  During pipeline operation, the \fpsf should
produce a small amount of ASCII text describing 

\begin{itemize}
  \item its progress in correcting the photometry postage stamps
  \item a list of those bright objects chosen to define the PSF
        (and those {\it not} chosen)
  \item a brief description of the PSF
\end{itemize}

  For debugging purposes, the module should produce a more verbose
text based upon the ``debug mode.''  Also, when debugging, 
it should print out values of the parameters as it reads them 
and the status of its memory-allocation attempts.

  The \fpsf module will not require much
memory.

{\it Note that this will not allow us to characterize the extended
wings of very bright stars, since the original postage stamps are
29x29 pixels.}
In order to model the wings, we have two options:
we can wait until the optical design is completed, 
attempt to calculate the form of the wings and 
make an analytic approximation;
or we can wait until the Test Year, look at actual
stellar images, and make the approximation at that time.
I suggest that we use some very simple analytic formula
for Level Zero, and substitute more accurate models into
later versions of the pipeline.

\subsection {Test Data Required}

  We will need to know what the derived PSF
{\it ought} to be, in order to verify that the module has 
done its job.  Therefore, all these products must be the result
of a simulation, rather than actual data from some telescope.
{\bf Creating this complete set of photometric data is a major
task that has not previously been identified.}

  As mentioned above, we cannot model the form of the extended
wings to stars until the entire optical system is finished,
and probably not until the Test Year.  However, we can make
a ``best guess'' for the Level Zero pipeline.

\subsection {Regression Testing}

  The simulated data set mentioned in the previous subsection 
will serve as a verification testbed for new versions of the
\fpsf .  We should save the module output from a
``correct'' version of the pipeline, and then compare that
output with results from new versions.  The comparison will 
require 

\begin{itemize}
  \item a set of ``fuzzy'' comparison tools for lists of numerical
        information (the same tools that are needed by many other
        pipeline modules, such as {\bf Measure Objects})
  \item a special set of tools for displaying and comparing
        {\it PSF} structures
\end{itemize}

If we run out of time and/or manpower, it is {\it possible} that
comparing the ASCII output which describes the PSF would suffice,
especially if the code which creates the output can be directed
to produce numbers which have been rounded to some standard precision.

